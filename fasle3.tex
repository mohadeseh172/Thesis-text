\documentclass[12pt]{book}

\usepackage{hyperref}

\usepackage{amsthm,amssymb,amsmath}
\usepackage[top=40mm, bottom=30mm, left=25mm, right=35mm]{geometry}
\setlength{\textwidth}{15cm}
\setlength{\oddsidemargin}{2mm}
\renewcommand{\baselinestretch}{1.5}




\usepackage{xepersian}
\setcounter{secnumdepth}{4}
\setcounter{tocdepth}{4}
\settextfont{B Nazanin}



\begin{document}

\chapter{تعاریف و مفاهیم مقدماتی}\label{chp1}
\section{مقدمه}



در این مقاله به بررسی مسئله زیر پرداخته می‌شود.
فرض کنید تعدادی هدف که اجسام ساکن هستند ، وجود دارم و میخواهیم از آن ها تصویر برداری کنیم . مسئله ما مسئله
\lr{coverage path planning} 
 است که باید تمام محیط داده شده را رویت کند و برای اینکه تمام آن رویت شود ، یک مسیر حرکت برای روبات داده میشود ، که اگر روی آن حرکت کند تمام محیط کاور میشود . این محیط باید توسط روبات گه در اینجا پهپاد های بدون سرنشین هستند و زیر آنها یک دوربین با زاویه دید 90 درجه قرار دارد، پوشش داده میشود. همان طور که در تصویر مشاهده میکنید، تصویر و محدوده دید را میتوان توسط یک مخروط مدل کرد . یک ورژن از مسئله
\lr{CPP} 
، به جای محیط یک تعداد هدف و نقطه به ما داده میشود ، این تعداد نقطه حتما باید پوشش داده شود. که جای این اهداف ثابت و مشخص است.
اما ممکن است اجسام دیگری مانع دیده شدن آن اهداف شوند ، مثلا فرض کنید اگر یک توپ زیر یک درخت باشد و هدف پوشش توپ باشد، پهپاد نمیتواند تصویر توپ را از بالا بدست آورده و باید از زاویه دیگر و از پهلو تصویر مورد نظر خود را بدست آورد. 
از آنجایی که روبات با توجه به محدودیت انرژی ای که دارد، فقط میتواند در زمان محدودی ، پرواز کند، هدف پیدا کردن کوتاه ترین مسیر ممکن برای حرکت پهپاد است به طوریکه پهپاد بتواند از تمام اهداف تصویر برداری کند و به نقطه اول خود بازگردد.
همانطور که گفته شده محدوده ی دید روبات با توجه به محدوده دید دوربین شبیه به یک مخروط است. حال اگر فرض کنید جای جسم و دوربین عوض شود، یک مخروط برعکس بدست می آید که پهپاد در محدوده دید از سمت جسم قرار دارد و درون مخروط برعکس است. میتوان گفت که اگرپهپاد در محدوده ی درون مخروط برعکس قرار بگیرد، همچنان میتواند جسم را رویت کند.
حال مسئله تبدیل میشود به صورت زیر:
پیدا کردن کوتاه ترین مسیری که بتوان پهپاد از آن عبور کند به طوریکه از تمام مخروط های برعکس حداقل یک بار عبور کند و به نقطه ابتدایی بازگردد. این مسئله را میتوان معادل مسئله تعمیم یافته
\lr{tsp} 
 دانست که در آن نقاط با مناطق جایگزین میشوند و هدف یافتن یک تور با کمترین هزینه است که باید از هر یک از مناطق یک بار بازدید کند و چون در این مسئله مناطق به صورت مخروط هستند مسئله به
\lr{ Cone TSPN} 
تعمیم میابد.
ینابراین میتوان مسئله را به صورت زیر فرمول بندی کرد:
مجموعه مخروط‌‌‌ها را به صورت $\mathrm{C}=\left[C_{1}, \ldots, C_{n}\right]$
نام‌گذاری می‌کنیم. همچنین مجموعه محل تقاطع مخروط‌‌‌ها با زمین(رئوس مخروط) را به صورت $\mathrm{X}=\left[X_{1}, \ldots, X_{n}\right]$
و مجموعه محور مخروط‌‌‌ها را به صورت $\mathrm{A}=\left[\vec{a}_{1}, \ldots, \vec{a}_{n}\right]$
نام‌گذاری می‌کنیم. همچنین اندازه زاویه راس مخروط‌‌‌ها مقدار ثابت ۲$
\alpha$
در نظر می‌گیریم (به اندازه زاویه دید دوربین). زاویه محور هر مخروط 
\lr{i} 
 با زمین نیز برابر با $
\varepsilon_{i}$
می‌باشد. حال مساله،‌‌ پیدا کردن کوتاه‌ترین مسیری است که بتواند از مجموعه متناهی از مخروط‌‌‌های زاویه‌دار عبور کند به طوری که از هر مخروط حداقل یک بار عبور کرده باشد.
ایسلر و همکاران برای پیدا کردن کوتاه‌ترین مسیر برای عبور از تعداد متناهی مخروط با زاویه و دیدن آن‌‌‌ها، یک الگوریتم تقریبی با تقریب زیر داده‌اند:
\begin{equation}
O\left(\frac{1+\tan \alpha}{1-\tan \epsilon \tan \alpha}\left(1+\log \frac{\max (H)}{\min (H)}\right)\right)
\end{equation}
هدف ما در این تحقیق، بررسی و بهبود الگوریتم حل این مساله می‌باشد.

روش کار آنها به صورت زیر میباشد:
1-
ابتدا مسئله را در حالتی که ارتفاع مخروط ها ثابت بوده و زاویه آنها برابر است و و از هم جدا و مستقل هستند را بررسی میکنیم:

1-1-	ابتدا حد بالایی برای فضایی که روبات بتواند در آن محدوده حرکت کند برای حالت 1 بدست میاوریم. برای این کار فضایی بین دو مخروط در حالتی که ارتفاع یکسان و زاویه یکسان دارند را میتوان به صورت زیر محاسبه کرد. اگر دو مخروط به شکل زیر باشند، فضای حرکت روبات را میتوان به سه مجموعه فضا زیر تقسیم کرد:
1.بخش بیرونی اولین و آخرین مخروط 
2. نیم استوانه ی قسمت بالایی 
3. گوه حاصل از قسمت از منطقه میانی 
که با جمع این سه فضا میتوان فهمید برای فضای مینکوفسکی حاصل ، میتوان به صورت زیر کران بالا بدست آورد.
\begin{equation}
f\left(T_{G}^{*}, \hat{h}\right) \leqslant L^{*} \hat{h}^{2} \tan \alpha\left(1+\frac{5 \pi}{6} \tan \alpha\right)
\end{equation}
1-2-	حال برای فضا حد پایین را بدست می آوریم:
فرض کنید در این حالت میخواهیم فضای قابل عبور بین دو مخروط را بدست آوریم که کمترین مقدار را دارند ، در این حالت مخروط ها باید عمودی بوده و مانند شکل زیر با هم تقاطع داشته باشند. در این حالت یکی از مخروط ها را گویی به ارتفاع he برده و باید اندازه محدوده هاشور خورده را بدست آوریم که به شکل زیر محاسبه میشود:
\begin{equation}
\begin{aligned}
V_{\text {intersect }} & \geq 4 \int_{\frac{h_{i}}{2}}^{h_{i}} \int_{0}^{\left.\sqrt{\left(\frac{r_{z}}{h_{i}}\right.}\right)^{2}-\frac{r_{i}^{2}}{4}}\left(\sqrt{\left(\frac{r_{i} z}{h_{i}}\right)^{2}-x^{2}}-\frac{r_{i}}{2}\right) \\
d x d z & \approx 0.2305 h_{i} r_{i}^{2} \\
& f\left(T_{G}^{*}, \hat{h}\right) \geqslant C_{v} \tan ^{2} \alpha \sum_{i} \in H h_{i}^{3}
\end{aligned}
\end{equation}
که در این رابطه
$ Cv=0.23 $ 
میباشد.

اگر یک مخروط با زاویه ε جابه جا شود به مکانی در همان راستا و با همان زاویه قبل ، میزان تغییر ارتفاع حاصله برابر است با
$r \tan \varepsilon$
 که از تساوی دو مثلث 1 و 2 بدست می آید.
حال با توجه به نکته فوق حد پایین در حالتی که مخروط ها زاویه داشته باشند را میتوان از رابطه زیر بدست آورد :
\begin{equation}
h_{i}-h_{e}=h(1-\tan \alpha \tan \varepsilon)
\end{equation}
و داریم:
\begin{equation}
\begin{gathered}
r_{\epsilon}=h \tan \alpha(1-\tan \alpha \tan \varepsilon) \\
f\left(T_{G}^{*}, \hat{h}\right) \geq C_{v}(1-\tan \alpha \tan \varepsilon)^{3} \tan \alpha^{2} \sum_{h_{i} \in H} h_{i}^{3}
\end{gathered}
\end{equation}
که در این حالت مخروط ها را عمودی فرض شده اند که با هم تقاطع دارند، مانند لم بالا که با کم کردن فضای مشترک که از رابطه لم دو بدست آمد.
حال در این صورت اگر به روش الگوریتم زیر (الگوریتم 1) تور را بدست آوریم؛ رابطه اندازه تور بدست آمده با تور بهینه از رابطه فوق بدست میآید:
\begin{equation}
L \leq(1+\beta)\left(2 h_{t}+\frac{L^{*}+2 n \tan \alpha}{\cos \varepsilon} \operatorname{mean}(H)\right)
\end{equation}
حال با توجه به رابطه بدست آمده ، زمان الگوریتم تقریبی برابر است:
\begin{equation}
(1+\beta)\left(\frac{1}{\cos \varepsilon}\left(\frac{\widehat{h}}{\operatorname{mean}(H)}\right)^{2}\left(1+\frac{5 \pi}{6} \tan \alpha\right)\right)
\end{equation}

2-
در این فصل مخروط های جدا از هم با زاویه های یکسان و یکتا که میتوانند ارتفاع های مختلف داشته باشند را بررسی میکنیم .
در این حالت چون الگوریتم قبلی کارآمدی کمتری دارد یک تغییری در آن ایجاد میکنیم و طبق الگوریتم 2 ابتدا ارتفاع ها که در بازه بین
$ h_{\min }  $
 و
$h_{\max }$
قرار دارد را تقسیم بندی میکنیم به بازه های 
$\left[2^{i} h_{\min }, 2^{i+1} h_{\min }\right)$
و به این ترتیب به تعداد
$\log _{2} h_{\max }-h_{\min }$
 بازه خواهیم داشت. حالا در هر بازه تعداد محدودی مخروط وجود دارد که ارتفاع آن ها در این بازه قرار بگیرد . برای هر بازه با تعداد خاص و محدود مخروط ، از الگوریتم 1 استفاده کرده و تور را بدست می آوریم ، سپس ، تور های بدست آمده در هر بازه ی ارتفاعی را به هم متصل کرده تا تور نهایی بدست آید.
قضیه 7.
این کار باعث میشود زمان انجام الگوریتم تقریبی فوق به مقدار زیر تغییر کند:
\begin{equation}
(1+\beta)\left(\frac{1}{\cos \varepsilon}\left(1+\log _{2} \frac{\widehat{h}}{h_{\min }}\right)\left(1+\frac{5 \pi}{6} \tan \alpha\right)\right)
\end{equation}

3-
مخروط هایی که لزوما از هم جدا نبوده و جهت دهی یکسان دارند یعنی زاویه یکسان دارند.
در این بخش به بررسی مخروط هایی میپردازیم که لزوما جدا از هم نبوده اما زاویه یکسانی دارند . برای بررسی این حالت ما از روش حالت قبل استفاده میکنیم با این تفاوت که ابتدا می آییم ، مسئله را به نوعی به حالت قبل تبدیل میکنیم تا برای تصادم ها (محل تقاطع مخروط ها) هم بتوانیم از روش قبل استفاده کنیم . برای این کار ، ابتدا می آِییم مجموعه ماکسیمال مخروط های جدا از هم را بدست می آوریم . این مجموعه مخروط با تمام مجموعه مخروط ها تقاطع دارند . حال استراتژی این است که الگورتیم قبل را برای این مجموعه پیاده سازی کینم ، سپس به تور بدست آمده، به ازای هر نقطه یک دیتور  اضافه کنیم تا مطمئن شویم تور حاصله تمام مخروط هارا پوشش میدهد.

لم 8- اگر مجموعه
 MIS
 را بدست آوریمو بعد با استفاده از الگوریتم 6 ، اگر با ضریب
$k=\left[\frac{2}{1-\tan \alpha \tan \varepsilon}\right]$
، دیتوری به تورمان اضافه کنیم ، طول دیتور حاصل کمتر مساوی 
$(8 \mathrm{k} \pi+4) h_{t} \tan \alpha$

خواهد بود و از همه ی مخروط ها هم بازدید خواهد کرد. اثبات این مسئله به این صورت است که ماکسیمم فاصله بین دو 
$\overrightarrow{a_{\imath}}$
و
$\overrightarrow{a_{\jmath}}$
که 
$C_{i} \in M I S$
و 
$C_{j} \notin M I S$
هست، و این دو با هم تقاطع دارند برابر است با
$\varepsilon h_{t} \tan \alpha$
چون 
$2 h_{t} \leq h_{i}$
و اگر مخروط ها زاویه 
$\varepsilon$
داشته باشند ، با استفاده از لم 3 ، وضعیت این مخروط ها همانند وضعیتی است که گویی یکی از آنها به اندازه ی 
$h_{e}=h_{i} \tan \alpha \tan \varepsilon$
بالا آورده شده باشد . پس بیشترین فاصله بین 
$\overrightarrow{a_{\imath}}$
و 
$\overrightarrow{a_{\jmath}}$
برابرست با 
$\varepsilon h_{t} \tan \alpha$
.
اگر مخروط بالا آمده 
$C_{j} $
باشد و مخروط حاصل از این انتقال را 
$C_{j}^{\prime}$
بنامیم، اندازه قطر این مخروط در ارتفاع
$h_{t} $
برابر میشود با 
$d_{j e}=2\left(h_{t}-h_{e}\right) \tan \alpha$
که با این کار میتوانیم پوشش 
$\overrightarrow{a_{\imath}}$
توسط
$\overrightarrow{a_{\jmath}^{\prime}}$
را با اضافه کردن یک دیتور به قطر
$d_{j e}$
تضمین کنیم. اندازه ی کلی تور بدست آمده با اعمال این تغییرات کمتر از
$8 \pi k h_{t} \tan \alpha+4 h_{t} \tan \varepsilon$
میشود که مقدار
 k 
در آن ثابت بوده و برابر است با
$k=\left[\frac{4 h_{t} \tan \alpha}{2\left(h_{t} h_{e}\right) \tan \alpha}\right]=\left[\frac{2}{1+\tan \alpha \tan \varepsilon}\right]$
. با این حساب میتوانیم دریابیم که :
قضیه 9:
از آنجایی که :

\begin{equation}
L \leq(1+\beta)\left(2 h_{t}+\frac{L^{*}}{\cos \varepsilon}+2 \frac{\operatorname{mean}(H)+h_{t} k(8 \pi+4)}{\cos \varepsilon} \times \frac{L^{*} \hat{h}^{2}\left(1+\frac{5 \pi}{6} \tan \alpha\right)}{C_{\varepsilon} \operatorname{mean}(H)^{3}}\right)
\end{equation}

هست، پس میتوان نتیجه گرفت الگریتم تقریبی حاصل دارای ضریب تقریب زیر است:
\begin{equation}
(1+\beta)\left(\frac{18 k}{\cos \varepsilon}\left(1+\frac{5 \pi \tan \alpha}{6}\right)\left[1+\log _{2} \frac{\hat{h}}{h_{\min }}\right]\right)
\end{equation}
4-
مخروط هایی که لزوما جدا از هم نیستند و جهت گیری های متفاوتی دارند.
در این حالت مخروط ها را بر اساس زاویه ای که دارند به صورت زیر دسته بندی میکنیم:
\begin{equation}
\widehat{\varepsilon}_{l} \varepsilon_{j}=\theta \leq \frac{\alpha}{2}, \forall i, j \quad \theta+\alpha \in\left(0, \frac{\pi}{2}\right)
\end{equation}
برای مخروط های در هر دسته ، نشان میدهیم که برای هر دسته ، کافی است که یک دیتور خاص خودش را بزنیم و چون در این حالت ما زاوایای مختلف داریم ،
 ε 
را به صورت زیر تعریف میکنیم :
\begin{equation}
\boldsymbol{\epsilon}=\underset{a_{i}}{\arg \max } \vec{n} \angle \vec{a}_{i}
\end{equation}
لم 10.
فرض کنید دو مخروط متقاطع داریم که 
$C_{i} \in M I S$
و 
$C_{j} \notin M I S$
و با زاویه راس 
$\alpha$
، ارتفاع های راس
$h_{i} $
و
$h_{j} $
و زاویه 
$\varepsilon_{i}$
و
$\varepsilon_{j}$
و
$\widehat{\varepsilon}_{l} \varepsilon_{j}=\theta \leq \frac{\alpha}{2}, \forall i, j \quad \theta+\alpha \in\left(0, \frac{\pi}{2}\right)$
باشد، داریم، اگر ارتفاع پوشش 
$h_{t} \geq h_{i} \frac{\sin \theta}{\sin (2 \alpha+\theta)}+r \tan \varepsilon$
باشد، برای آنکه بتوانیم تضمین کنیم تور حاصله از
 MIS 
بدست آمده تمام مخروط هارا پوشش میدهد ، باید یک مسیر انحرافی و دور اضافی را به تور حاصله از الگوریتم 3 اضافه کنیم که در فاصله 
$4 h_{t} \tan \alpha+2\left(h_{t}-r \tan \varepsilon\right) \tan \alpha$
قرار دارد.





\end{document}